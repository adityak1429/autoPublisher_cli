\documentclass{article}
\usepackage[utf8]{inputenc}
\usepackage{hyperref}
\title{Microsoft Store App Update Automation: Technical Flow Documentation}
\author{Sai Aditya Kukudala}
\date{\today}

\begin{document}

\maketitle

\section{Introduction}
This document describes the technical flow for automating Microsoft Store app updates using the \texttt{autoPublisher} CLI. It covers the API endpoints used, the structure of payloads, and the sequence of operations performed to update an app submission.

\section{Update Flow Overview}
The update process for a Microsoft Store app via automation consists of the following high-level steps:
\begin{enumerate}
    \item Authenticate and obtain an access token.
    \item Delete any existing in-progress submission.
    \item Create a new submission.
    \item Prepare and update submission metadata.
    \item Upload package and media assets.
    \item Commit the submission.
    \item Poll for submission status.
\end{enumerate}

\section{APIs Used}
All API calls are made to the Microsoft Store Dev Center API (\url{https://manage.devcenter.microsoft.com/v1.0/my/}). Authentication uses Azure AD OAuth2.

\begin{itemize}
    \item \textbf{Token Endpoint:} \url{https://login.microsoftonline.com/\{tenantId\}/oauth2/token}
    \item \textbf{Get App Data:} \texttt{GET applications/\{productId\}}
    \item \textbf{Delete Submission:} \texttt{DELETE applications/\{productId\}/submissions/\{submissionId\}}
    \item \textbf{Create Submission:} \texttt{POST applications/\{productId\}/submissions}
    \item \textbf{Get Submission Metadata:} \texttt{GET applications/\{productId\}/submissions/\{submissionId\}}
    \item \textbf{Update Submission Metadata:} \texttt{PUT applications/\{productId\}/submissions/\{submissionId\}}
    \item \textbf{Commit Submission:} \texttt{POST applications/\{productId\}/submissions/\{submissionId\}/commit}
    \item \textbf{Get Submission Status:} \texttt{GET applications/\{productId\}/submissions/\{submissionId\}/status}
\end{itemize}

\section{Authentication}
OAuth2 client credentials flow is used. The following payload is sent to the token endpoint:
\begin{verbatim}
grant_type=client_credentials
client_id=<client-id>
client_secret=<client-secret>
resource=https://manage.devcenter.microsoft.com
\end{verbatim}
The resulting \texttt{access\_token} is used as a Bearer token in all subsequent API requests.

\section{Submission Metadata Payload}
The main payload for updating a submission is a JSON object matching the \texttt{ApplicationSubmissionRequest} schema. Only a subset of fields are required for updates. Example structure:
\begin{verbatim}
{
  "ApplicationCategory": "HealthAndFitness",
  "Pricing": { ... },
  "Visibility": "Public",
  "TargetPublishMode": "Immediate",
  "TargetPublishDate": "1601-01-01T00:00:00Z",
  "Listings": {
    "en-us": {
      "BaseListing": {
        "description": "...",
        "Images": [
          {
            "fileName": "Screenshot1.png",
            "FileStatus": "PendingUpload",
            "ImageType": "Screenshot"
          }
        ],
        ...
      }
    }
  },
  "ApplicationPackages": [
    {
      "fileName": "myapp.msixbundle",
      "FileStatus": "PendingUpload"
    }
  ],
  "Trailers": [
    {
      "VideoFileName": "Trailer_demo.mp4",
      "TrailerAssets": {
        "en-us": {
          "title": "demo",
          "imageList": [
            {
              "fileName": "TrailerImage_demo.png",
              "description": null
            }
          ]
        }
      }
    }
  ]
}
\end{verbatim}
\textbf{Note:} Fields like \texttt{ApplicationPackages} and \texttt{Images} must be managed carefully. Existing assets are marked \texttt{PendingDelete}, and new assets are added with \texttt{PendingUpload}.

\section{Sequence of API Calls}
\begin{enumerate}
    \item \textbf{Authenticate:} Obtain access token.
    \item \textbf{Get Current Submission:} \texttt{GET applications/\{productId\}} to check for pending submissions.
    \item \textbf{Delete Submission:} If a pending submission exists, \texttt{DELETE} it.
    \item \textbf{Create Submission:} \texttt{POST} to create a new submission.
    \item \textbf{Get Metadata:} \texttt{GET} submission metadata for the new submission.
    \item \textbf{Update Metadata:} \texttt{PUT} updated metadata payload.
    \item \textbf{Upload Assets:} Use the \texttt{fileUploadUrl} from metadata to upload the package via Azure Blob Storage.
    \item \textbf{Commit Submission:} \texttt{POST} to commit the submission.
    \item \textbf{Poll Status:} \texttt{GET} status until published or failed.
\end{enumerate}

\section{Error Handling}
Common errors include:
\begin{itemize}
    \item \textbf{InvalidState:} Submission is not in a modifiable state. Solution: Delete and recreate the submission.
    \item \textbf{Schema Errors:} Payload does not match expected schema. Solution: Validate JSON structure and field types.
    \item \textbf{Authentication Errors:} Invalid credentials or expired token.
\end{itemize}

\section{References}
\begin{itemize}
    \item \href{https://learn.microsoft.com/en-us/windows/uwp/publish/app-submission-api-overview}{Microsoft Store Submission API Overview}
    \item \href{https://learn.microsoft.com/en-us/windows/uwp/publish/create-app-submission}{Create and manage app submissions}
\end{itemize}

\end{document}